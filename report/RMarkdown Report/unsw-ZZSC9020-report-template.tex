\documentclass[mstat,12pt,a4paper]{unswthesis}

\usepackage{color}
\usepackage{fancyvrb}
\newcommand{\VerbBar}{|}
\newcommand{\VERB}{\Verb[commandchars=\\\{\}]}
\DefineVerbatimEnvironment{Highlighting}{Verbatim}{commandchars=\\\{\}}
% Add ',fontsize=\small' for more characters per line
\usepackage{framed}
\definecolor{shadecolor}{RGB}{248,248,248}
\newenvironment{Shaded}{\begin{snugshade}}{\end{snugshade}}
\newcommand{\AlertTok}[1]{\textcolor[rgb]{0.94,0.16,0.16}{#1}}
\newcommand{\AnnotationTok}[1]{\textcolor[rgb]{0.56,0.35,0.01}{\textbf{\textit{#1}}}}
\newcommand{\AttributeTok}[1]{\textcolor[rgb]{0.13,0.29,0.53}{#1}}
\newcommand{\BaseNTok}[1]{\textcolor[rgb]{0.00,0.00,0.81}{#1}}
\newcommand{\BuiltInTok}[1]{#1}
\newcommand{\CharTok}[1]{\textcolor[rgb]{0.31,0.60,0.02}{#1}}
\newcommand{\CommentTok}[1]{\textcolor[rgb]{0.56,0.35,0.01}{\textit{#1}}}
\newcommand{\CommentVarTok}[1]{\textcolor[rgb]{0.56,0.35,0.01}{\textbf{\textit{#1}}}}
\newcommand{\ConstantTok}[1]{\textcolor[rgb]{0.56,0.35,0.01}{#1}}
\newcommand{\ControlFlowTok}[1]{\textcolor[rgb]{0.13,0.29,0.53}{\textbf{#1}}}
\newcommand{\DataTypeTok}[1]{\textcolor[rgb]{0.13,0.29,0.53}{#1}}
\newcommand{\DecValTok}[1]{\textcolor[rgb]{0.00,0.00,0.81}{#1}}
\newcommand{\DocumentationTok}[1]{\textcolor[rgb]{0.56,0.35,0.01}{\textbf{\textit{#1}}}}
\newcommand{\ErrorTok}[1]{\textcolor[rgb]{0.64,0.00,0.00}{\textbf{#1}}}
\newcommand{\ExtensionTok}[1]{#1}
\newcommand{\FloatTok}[1]{\textcolor[rgb]{0.00,0.00,0.81}{#1}}
\newcommand{\FunctionTok}[1]{\textcolor[rgb]{0.13,0.29,0.53}{\textbf{#1}}}
\newcommand{\ImportTok}[1]{#1}
\newcommand{\InformationTok}[1]{\textcolor[rgb]{0.56,0.35,0.01}{\textbf{\textit{#1}}}}
\newcommand{\KeywordTok}[1]{\textcolor[rgb]{0.13,0.29,0.53}{\textbf{#1}}}
\newcommand{\NormalTok}[1]{#1}
\newcommand{\OperatorTok}[1]{\textcolor[rgb]{0.81,0.36,0.00}{\textbf{#1}}}
\newcommand{\OtherTok}[1]{\textcolor[rgb]{0.56,0.35,0.01}{#1}}
\newcommand{\PreprocessorTok}[1]{\textcolor[rgb]{0.56,0.35,0.01}{\textit{#1}}}
\newcommand{\RegionMarkerTok}[1]{#1}
\newcommand{\SpecialCharTok}[1]{\textcolor[rgb]{0.81,0.36,0.00}{\textbf{#1}}}
\newcommand{\SpecialStringTok}[1]{\textcolor[rgb]{0.31,0.60,0.02}{#1}}
\newcommand{\StringTok}[1]{\textcolor[rgb]{0.31,0.60,0.02}{#1}}
\newcommand{\VariableTok}[1]{\textcolor[rgb]{0.00,0.00,0.00}{#1}}
\newcommand{\VerbatimStringTok}[1]{\textcolor[rgb]{0.31,0.60,0.02}{#1}}
\newcommand{\WarningTok}[1]{\textcolor[rgb]{0.56,0.35,0.01}{\textbf{\textit{#1}}}}




%%%%%%%%%%%%%%%%%%%%%%%%%%%%%%%%%%%%%%%%%%%%%%%%%%%%%%%%%%%%%%%%%%
% 
% OK...Now we get to some actual input.  The first part sets up
% the title etc that will appear on the front page
%
%%%%%%%%%%%%%%%%%%%%%%%%%%%%%%%%%%%%%%%%%%%%%%%%%%%%%%%%%%%%%%%%%

\title{A Data Science Approach to Forecast Electricity Consumption in
Australia}

\authornameonly{David Valido Ramos (z5516338), Katelyn Kemp
(z5459347), Nick Mutton (z5549371), Senarath Seelanatha
(z5595581), Shanjay Perinpanathan (z5339723), Waseem Alashqar
(z5514810). }

\author{\Authornameonly}

\copyrightfalse
\figurespagefalse
\tablespagefalse

%%%%%%%%%%%%%%%%%%%%%%%%%%%%%%%%%%%%%%%%%%%%%%%%%%%%%%%%%%%%%%%%%
%
%  And now the document begins
%  The \beforepreface and \afterpreface commands puts the
%  contents page etc in
%
%%%%%%%%%%%%%%%%%%%%%%%%%%%%%%%%%%%%%%%%%%%%%%%%%%%%%%%%%%%%%%%%%%


\input{header.tex}

\begin{document}

\beforepreface

%\afterpage{\blankpage}


%\afterpage{\blankpage}

% Abstract

\prefacesection{Abstract}



%\afterpage{\blankpage}


\afterpreface





%%%%%%%%%%%%%%%%%%%%%%%%%%%%%%%%%%%%%%%%%%%%%%%%%%%%%%%%%%%%%%%%%%
%
% Now we can start on the first chapter
% Within chapters we have sections, subsections and so forth
%
%%%%%%%%%%%%%%%%%%%%%%%%%%%%%%%%%%%%%%%%%%%%%%%%%%%%%%%%%%%%%%%%%%



%%%%%%%%%%%%%%%%%%%%%%%%%%%%%%%%%%%%%

%\afterpage{\blankpage}


\chapter{Introduction}\label{introduction}

The ability to accurately forecast electricity demand is vital for
future planning by informing decision making that will, among other
things, manage costs, and ensure a reliable power supply. According to
the client for this project, one of the greatest drivers of variability
in electricity demand is the weather, therefore it is important
forecasts consider this to improve accuracy. A summary of literature
related to electricity forecasting, including incorporating temperature
effects is presented in section 2.

\bigskip

The client has provided data related to electricity demand and forecast
demand in NSW (sourced from AEMO's Market Management System database),
along with air temperature data for Bankstown (sourced from the
Australian Data Archive for Meteorology) that will form the basis for
this project's forecast modelling.

\bigskip

Initial team analysis of forecast data against actual demand data
revealed current electricity forecasts struggle to perform during hotter
temperatures. Furthermore, there is expected to be an increase in
extreme temperature events due to climate change (BOM, 2024).

\bigskip

The goal of this project is to improve electricity demand forecasting in
periods of extreme temperatures, especially for extended periods of
extreme temperature. The analysis will involve short-term forecasting
(up to 40 hours) to improve on AEMO's forecast specifically, through
greater consideration of temporal weather effects including extreme high
or low temperatures (e.g.~heatwaves).

\bigskip

Initially, time series modelling will be used to understand temporal
effects and behaviours of temperature and electricity demand. Other
machine learning techniques will also be explored, evaluated and
implemented as required. This project plan details software, methods,
tasks and team roles.

\chapter{Literature Review}\label{literature-review}

\section{Forecasting electricity
demand}\label{forecasting-electricity-demand}

\bigskip

Energy forecasts play a crucial role in planning and maintaining the
energy sector, and ensuring forecasting accuracy helps to manage
imbalances in energy production and consumption, reduce power system
costs and improve operational safety (Koukaras at al., 2024). Energy
forecasting therefore has a broad impact on a wide variety of
stakeholders including residential customers, power generators,
retailers, traders, industrial and commercial customers, system
operators, and financial investors (Ghalehkhondabi et al., 2016).

\bigskip

There are many risks in inaccurate energy forecasting. Over forecasting
has cost and resource implications for providers, as well as
environmental impacts. Under forecasting can cause outages, as well as
having down the stream increased costs from inconsistent supply
(Suganthi and Samuel, 2012). Shortages are also linked to political
instability (Rakpho and Yamaka, 2021).

\bigskip

In the modern world where energy consumption is almost continuous,
energy forecasting has become quite complex. There is no universal
method of forecasting energy demand, and it is the data and the business
need that tends to determine which technique is most useful (Pinheiro,
Madeira \& Francisco, 2023). Despite this, the forecast interval, which
often indicates the goal of the forecast, also influences the kinds of
models used.

\bigskip

Forecasting models are typically categorised into short-, medium-, and
long-term, and while there is not a unanimous definition of what
constitutes these time periods, researchers generally agree that
short-term is a few minutes up to a few days (Ahmad and Chen, 2018) or
two weeks (Klyuev et al., 2022), medium-term as one month to one year,
and long-term as one year to ten years (Ahmad and Chen, 2018). The
Australian Energy Market Operator (AEMO) has a more general
classification with short-term as up to five years, and long term as
longer than this (AEMO, 2022).

\bigskip

Short-term intervals tend to be the most accurate and are important for
the effective management of electricity demand and help to reduce peak
loads (need reference). Short-term forecast methods can be broadly
categorised into two categories -- mathematical algorithms such as
time-series analysis and logistic regression, and artificial
intelligence (AI) algorithms such as machine learning, deep learning and
ensemble learning models (Deng et al., 2022). For short-term
forecasting, AI methods are becoming more popular as they can consider
the non-linear nature of power demand. Short term forecasting is also
generally more interested in the accuracy of the forecast rather than
the interpretability of the results which makes these `black box'
approaches appropriate (Klyuev et al., 2022). Other studies have found
that machine learning models tend to outperform traditional models such
as ARIMA in short-term forecasting (Divina et al., 2019).

\bigskip

Medium- and long-term forecasting supports the planning and maintenance
of the electrical network such as smart grid eco-systems (Ahmad \& Chen,
2018). Furthermore, long-term forecasting is more strategic and is
necessary for the development of energy systems, planning capital
construction at production or infrastructure facilities (Klyuev et al.,
2022). These forecast intervals typically use econometric models, system
dynamics, and grey prediction, with a focus on policy adjustments,
economic indicators (such as GDP and CPI), and population trends
(Mystakidis et al., 2024).

\section{Weather in forecasting electricity
demand}\label{weather-in-forecasting-electricity-demand}

Temperature is a primary driver of electricity demand, shaping heating
and cooling loads that dictate energy consumption. Research consistently
identifies it as the dominant weather factor in electricity demand
prediction, especially during peak periods. Liu et al.~(2021)
demonstrate that extreme temperatures lead to increased residential
electricity consumption, finding that for each additional day in which
the mean temperature exceeds 30 °C, there is an 16.8\% increase in
monthly residential electricity consumption. Similarly, for each
additional day below -6 °C there is a 6\% increase in monthly
residential electricity consumption. This underscores temperature's
critical role in accurate demand forecasting, as it directly influences
consumption patterns during extreme events.

\bigskip

Extreme temperatures can lead to significant errors in electricity
demand forecasts, often underestimating demand. During Winter Storm Uri
in Texas in February 2021 (Añel, 2024), minimum extreme cold
temperatures of --34 °C and high winds of 260 km/h impacted 170 million
people. Due to this extreme weather event, electricity demand
unexpectedly increased from 40 GW to over 70 GW, resulting on blackouts
that affected more than 4 million people. The economic cost of the power
outages and disruption has been estimated between 26.1 and 130 billion
U.S. dollars.

\section{Defining extreme
temperature}\label{defining-extreme-temperature}

There are various approaches to define extreme weather events. The
Intergovernmental Panel on Climate Change (IPCC, 2021) uses a
percentile-based definition, which ensures a global standardised
approach that determines the relative threshold of extreme weather
events by region. An extreme heat event is recorded if the maximum
temperature during a day is higher than the 90th percentile of
historical weather records. Similarly, an extreme cold event is recorded
if the minimum temperature during a day is lower than the 10th
percentile of historical weather records. Other methods involve an
absolute threshold (e.g.~35 °C is a hot day).

\bigskip

Other weather variables, particularly humidity and ``feels like''
temperature, enhance forecasting accuracy. Maia-Silva et al.~(2020)
found that using humidity-related measures, such as dew point and heat
index, improves prediction accuracy, especially in high-energy-consuming
regions, with improvements up to 8-9\%. This highlights the need to
consider composite weather indices, as air temperature alone
underestimates demand during humid heatwaves by as much as 10-15\%.

\bigskip

AEMO uses historical weather data from BOM for training their demand
forecasting models (AEMO, 2024 Forecasting Assumptions). A weather
station per region is selected based on data availability and
correlation with regional consumption or demand. The Bureau of
Meteorology (BOM) defines a heatwave as maximum temperatures unusually
hot for over 3 days compared to local climate, using the Excess Heat
Factor (EHF) for monitoring. The EHF combines the observed temperature
in the past month, and the average temperature for a 3-day period to
measure its severity.

\section{Modelling electricity demand in extreme
weather}\label{modelling-electricity-demand-in-extreme-weather}

We have narrowed our focus on a modification of short-term energy demand
forecasts. Based on the literature, it appears that it is best suited
for reducing demand-forecast variability. Temperature forecasts are
reasonably accurate for the forward 1-3 days with diminishing accuracy
up to eight days (Floehr, 2010), and energy suppliers have means of
modulating their energy outputs based on short-term forecasts (Hydro,
2020) (IEA, 2023). Short-term energy models can be effectively
categorised into two groups: those that prioritise explainability, and
those that focus on forecast accuracy.

\bigskip

Explainable models consist mostly of traditional statistical and
econometric models, such as regression (Papalexopoulos and Hesterberg,
1990) (Ertuğrul, Tekin and Tekin, 2020), time-series such as ARIMA
(Tarmanini et al., 2023) (Ediger and Akar, 2007), as well as decision
trees (Kopyt et al., 2024) (Wang et al., 2018). They also have natural
extrapolations to medium-to-long term models, that are also
econometric-based due to their relationship with longitudinal factors
such as policy changes, modifications to the energy grid, or economic
factors (such as GDP and population) (Ardakani and Ardehali, 2014).

\bigskip

Models that focus on forecast accuracy are generally black box machine
learning models such as Neural Networks (Manno, Martelli and Amaldi,
2022) (Kuo and Huang, 2018), Support Vector Machines (Ahmad et al.,
2014) (Ahmad et al., 2020), and ensemble methods, such as Random Forests
(Divina et al., 2019) and XGBoost (Abbasi et al., 2019).

\bigskip

However, the best performing models tend to be some form of hybrid model
using a combination of either explainable or machine learning models,
such as NN--ARIMA or CNN-LTSM due to their stability and potential to
reduce overfitting (Deng et al., 2022). This feature gives credence to
our supplementation of AEMO's forecast to improve extreme temperature
forecasts. Hybrid models have also been shown to produce superior
forecasts during temporal extreme temperature events (Phyo and Byun,
2021).

\bigskip

Short-term energy demand forecasting factors include current temperature
and other climate measures such as temperature forecast, humidity, dew
point, air pressure, windspeed, solar radiation etc (primarily focused
on capturing the ``feels like'' influence of weather), historical energy
demand (Suganthi and Samuel, 2012), and time factors such as time of
day, day of the week, month, and season (Boroojeni et al., 2017). For
extreme temperature, additional factors such as rolling average
temperature (Gutiérrez et al., 2013) and consecutive days over/under an
average temperature limit are also included (Zhang et al., 2022).
Unsupervised learning techniques, like k-means clustering, are also
utilised to reduce dimensionality and identify time associations (Singh
and Yassine, 2018).

\chapter{Material and Methods}\label{material-and-methods}

\section{Software}\label{software}

\begin{center}
\begin{tabular}{|l|l|p{12em}|}

\hline
\textbf{Software} & \textbf{Library(s)} & \textbf{Purpose} \\
\hline
\multirow{4}{4em}{Python} & Pandas & Reading, manipulating, cleaning\newline and analysing datasets. \\
\cline{2-3}
& Numpy & Manipulating data and mathematical calculations. \\
\cline{2-3}
& Matplotlib, Seaborn & Visualising data to understand trends and\newline patterns. \\
\cline{2-3}
& Scikit-learn & Implementing and evaluating machine\newline learning algorithms. \\
\hline
PowerBI & - & Summarising and visualising data. \\
\hline
RMarkdown & - & Writing final report. \\
\hline

\end{tabular}
\end{center}

\section{Description of the Data}\label{description-of-the-data}

Table xx describes the data that will be used in analysis. In addition
to the datafiles provided by the client, historical weather forecasts
including temperature, humidity and wind speed were sourced from
OpenWeather, a global company specializing in environmental data
products. Forecast weather data rather than actual weather data was used
to ensure inputs into the forecast model were realistic.

\begin{table}[H]
\centering
\begin{tabular}{|p{13em}|p{20em}|}

\hline
\textbf{Data} & \textbf{Description} \\

\hline
\textbf{Electricity demand}\newline Use for both training and testing models. & Electricity demand from 2010 to 2021. Well-structured and low complexity with no duplicates and no null values.\newline Variables: Date-time, totalDemand, regionID\newline
Format: CSV, Storage: Github, Size: 6 Mb, Rows: 196,513 \\

\hline
\textbf{Air temperature}\newline Use to train the behaviour of temperature as a forecasting input. & Provides air temperature data in Bankstown from 2010 to 2021. Well-structured with no null values. It is medium complexity due to uneven time increments and duplicate rows.\newline Variables: Date-time, location, temperature\newline
Format: CSV, Storage: Github, Size: 7 Mb, Rows: 220,326 \\

\hline
\textbf{Forecast demand} \newline Use as a baseline forecast model and improve it through the inclusion of more forecasting inputs. & Provides forecasted demand data from 2010 to 2021. Well-structured with no null values. It is high complexity due to uneven time increments and duplicate rows.\newline Variables: Date-time, forecastDemand, totalDemand, regionID, preDispatchSeqNo, periodID, lastChange\newline
Format: CSV, Storage: Github, Size: 722 Mb, Rows: 10,906,019 \\

\hline
\textbf{Forecast weather indicators} & Provides previous forecast weather data for Bankstown from October 7 2017.\newline Variables: temperature, humidity, wind speed\newline Format: CSV, Storage: XXX \\ 

\hline

\end{tabular}
\end{table}

\section{Pre-processing Steps}\label{pre-processing-steps}

What did you have to do to transform the data so that they become
useable?

\section{Data Cleaning}\label{data-cleaning}

Data was found to be complete for Electricity Demand and Air
temperature. Some forecast data were missing for forecast intervals
\textgreater12 hours. To ensure complete data was used, the forecast
model was trained and tested on 12 hour forecast intervals. There were
no missing values in the Forecast Weather Indicators data, however the
available data begins on October 7 2017.\\
\bigskip Consequently, the relevant data used to train and test the
forecast model was between October 7 2017 and 17 March 2021 with a 12
hour forecast interval. No further missing values were present in the
data. \bigskip Additional data cleaning steps performed on all datasets
are detailed below: \bigskip 1.Date/time variables were formatted
consistently (i.e.~d/m/y H:M).

2.Date/time variables were rounded to the nearest 30 minute increment to
provide consistent 30-minute intervals.

3.Duplicate date/time rows were removed to ensure each date/time row was
unique. \bigskip After each dataset was cleaned and checked, they were
merged into one clean dataset, joined on the unique date/time variable.

\section{Assumptions}\label{assumptions}

\section{Modelling Methods}\label{modelling-methods}

\chapter{Exploratory Data Analysis}\label{exploratory-data-analysis}

This section presents an exploratory analysis of the temperature,
forecasted demand, and actual electricity demand data. Emphasis is
placed on understanding how demand responds to temperature variations
and where forecast discrepancies are most pronounced. The data is
manipulated and visualised with Python.

\bigskip

\noindent Begin with importing the modules required and clean datasets.

\begin{Shaded}
\begin{Highlighting}[]
\ImportTok{import}\NormalTok{ pandas }\ImportTok{as}\NormalTok{ pd}
\ImportTok{import}\NormalTok{ numpy }\ImportTok{as}\NormalTok{ np}
\ImportTok{import}\NormalTok{ seaborn }\ImportTok{as}\NormalTok{ sns}
\ImportTok{import}\NormalTok{ matplotlib.pyplot }\ImportTok{as}\NormalTok{ plt}
\end{Highlighting}
\end{Shaded}

\begin{Shaded}
\begin{Highlighting}[]
\CommentTok{\#\#\# Temperature dataset}
\NormalTok{df\_temperature }\OperatorTok{=}\NormalTok{ pd.read\_csv(}\StringTok{\textquotesingle{}temperature\_nsw.csv\textquotesingle{}}\NormalTok{, }
\NormalTok{    names }\OperatorTok{=}\NormalTok{ [}\StringTok{\textquotesingle{}location\textquotesingle{}}\NormalTok{, }\StringTok{\textquotesingle{}date\_time\textquotesingle{}}\NormalTok{, }\StringTok{\textquotesingle{}temperature\textquotesingle{}}\NormalTok{], skiprows }\OperatorTok{=} \DecValTok{1}\NormalTok{)}
\NormalTok{df\_temperature.date\_time }\OperatorTok{=}\NormalTok{ pd.to\_datetime(df\_temperature.date\_time, }
    \BuiltInTok{format} \OperatorTok{=} \StringTok{"}\SpecialCharTok{\%d}\StringTok{/\%m/\%Y \%H:\%M"}\NormalTok{)}

\CommentTok{\#\#\# Demand dataset}
\NormalTok{df\_demand }\OperatorTok{=}\NormalTok{ pd.read\_csv(}\StringTok{\textquotesingle{}totaldemand\_nsw.csv\textquotesingle{}}\NormalTok{, }
\NormalTok{    names }\OperatorTok{=}\NormalTok{ [}\StringTok{\textquotesingle{}date\_time\textquotesingle{}}\NormalTok{, }\StringTok{\textquotesingle{}total\_demand\textquotesingle{}}\NormalTok{, }\StringTok{\textquotesingle{}region\_id\textquotesingle{}}\NormalTok{], skiprows }\OperatorTok{=} \DecValTok{1}\NormalTok{)}
\NormalTok{df\_demand.date\_time }\OperatorTok{=}\NormalTok{ pd.to\_datetime(}
\NormalTok{    df\_demand.date\_time, }\BuiltInTok{format} \OperatorTok{=} \StringTok{"}\SpecialCharTok{\%d}\StringTok{/\%m/\%Y \%H:\%M"}\NormalTok{)}

\CommentTok{\#\#\# Forecast dataset}
\NormalTok{df\_forecast }\OperatorTok{=}\NormalTok{ pd.read\_csv(}\StringTok{\textquotesingle{}forecastdemand\_nsw.csv\textquotesingle{}}\NormalTok{, }
\NormalTok{    names }\OperatorTok{=}\NormalTok{ [}\StringTok{\textquotesingle{}id\textquotesingle{}}\NormalTok{, }\StringTok{\textquotesingle{}region\_id\textquotesingle{}}\NormalTok{, }\StringTok{\textquotesingle{}period\_id\textquotesingle{}}\NormalTok{, }\StringTok{\textquotesingle{}forecast\_demand\textquotesingle{}}\NormalTok{, }
        \StringTok{\textquotesingle{}date\_time\_forecast\textquotesingle{}}\NormalTok{, }\StringTok{\textquotesingle{}date\_time\_prediction\textquotesingle{}}\NormalTok{], }
\NormalTok{    skiprows }\OperatorTok{=} \DecValTok{1}\NormalTok{)}
\NormalTok{df\_forecast.date\_time\_forecast }\OperatorTok{=}\NormalTok{ pd.to\_datetime(}
\NormalTok{    df\_forecast.date\_time\_forecast, }\BuiltInTok{format} \OperatorTok{=} \StringTok{"\%Y{-}\%m{-}}\SpecialCharTok{\%d}\StringTok{ \%H:\%M:\%S"}\NormalTok{)}
\NormalTok{df\_forecast.date\_time\_prediction }\OperatorTok{=}\NormalTok{ pd.to\_datetime(}
\NormalTok{    df\_forecast.date\_time\_prediction, }\BuiltInTok{format} \OperatorTok{=} \StringTok{"\%Y{-}\%m{-}}\SpecialCharTok{\%d}\StringTok{ \%H:\%M:\%S"}\NormalTok{)}
\end{Highlighting}
\end{Shaded}

\section{\texorpdfstring{Overview of datasets\textbf{(This section would
be going to previous
chapter)}}{Overview of datasets(This section would be going to previous chapter)}}\label{overview-of-datasetsthis-section-would-be-going-to-previous-chapter}

\subsection{Air temperature dataset}\label{air-temperature-dataset}

\begin{Shaded}
\begin{Highlighting}[]
\BuiltInTok{print}\NormalTok{(}\StringTok{"Locations = }\SpecialCharTok{\{\}}\StringTok{"}\NormalTok{.}\BuiltInTok{format}\NormalTok{(}\BuiltInTok{set}\NormalTok{(df\_temperature.location)))}
\BuiltInTok{print}\NormalTok{(}\StringTok{"Date Min = }\SpecialCharTok{\{\}}\StringTok{  |  Date Max = }\SpecialCharTok{\{\}}\StringTok{"}\NormalTok{.}\BuiltInTok{format}\NormalTok{(}
\NormalTok{    df\_temperature.date\_time.}\BuiltInTok{min}\NormalTok{(), df\_temperature.date\_time.}\BuiltInTok{max}\NormalTok{()))}
\BuiltInTok{print}\NormalTok{(}\StringTok{"Temp Min = }\SpecialCharTok{\{\}}\StringTok{  |  Temp Max = }\SpecialCharTok{\{\}}\CharTok{\textbackslash{}n}\StringTok{"}\NormalTok{.}\BuiltInTok{format}\NormalTok{(}
\NormalTok{    df\_temperature.temperature.}\BuiltInTok{min}\NormalTok{(), df\_temperature.temperature.}\BuiltInTok{max}\NormalTok{()))}
\BuiltInTok{print}\NormalTok{(df\_temperature.head())}
\BuiltInTok{print}\NormalTok{(}\StringTok{"}\CharTok{\textbackslash{}n}\StringTok{Rows = }\SpecialCharTok{\{\}}\CharTok{\textbackslash{}n}\StringTok{"}\NormalTok{.}\BuiltInTok{format}\NormalTok{(}\BuiltInTok{len}\NormalTok{(df\_temperature)))}
\BuiltInTok{print}\NormalTok{(}\StringTok{"Average Datetime frequency: }\SpecialCharTok{\{\}}\CharTok{\textbackslash{}n}\StringTok{"}\NormalTok{.}\BuiltInTok{format}\NormalTok{(df\_temperature.date\_time.diff().mean()))}
\BuiltInTok{print}\NormalTok{(}\StringTok{"NaN values }\CharTok{\textbackslash{}n}\StringTok{{-}{-}{-}{-}{-}{-}{-}{-}{-}{-}}\CharTok{\textbackslash{}n}\SpecialCharTok{\{\}}\StringTok{"}\NormalTok{.}\BuiltInTok{format}\NormalTok{(df\_temperature.isnull().}\BuiltInTok{sum}\NormalTok{().to\_string()))}
\end{Highlighting}
\end{Shaded}

\begin{verbatim}
## Locations = {'Bankstown'}
## Date Min = 2010-01-01 00:00:00  |  Date Max = 2021-03-18 00:00:00
## Temp Min = -1.3  |  Temp Max = 44.7
##     location           date_time  temperature
## 0  Bankstown 2010-01-01 00:00:00         23.1
## 1  Bankstown 2010-01-01 00:01:00         23.1
## 2  Bankstown 2010-01-01 00:30:00         22.9
## 3  Bankstown 2010-01-01 00:50:00         22.7
## 4  Bankstown 2010-01-01 01:00:00         22.6
## 
## Rows = 220326
## Average Datetime frequency: 0 days 00:26:45.453761488
## NaN values 
## ----------
## location       0
## date_time      0
## temperature    0
\end{verbatim}

\bigskip

\subsection{Electricity demand
dataset}\label{electricity-demand-dataset}

\begin{Shaded}
\begin{Highlighting}[]
\BuiltInTok{print}\NormalTok{(}\StringTok{"Regions = }\SpecialCharTok{\{\}}\StringTok{"}\NormalTok{.}\BuiltInTok{format}\NormalTok{(}\BuiltInTok{set}\NormalTok{(df\_demand.region\_id)))}
\BuiltInTok{print}\NormalTok{(}\StringTok{"Date Min = }\SpecialCharTok{\{\}}\StringTok{  |  Date Max = }\SpecialCharTok{\{\}}\StringTok{"}\NormalTok{.}\BuiltInTok{format}\NormalTok{(}
\NormalTok{    df\_demand.date\_time.}\BuiltInTok{min}\NormalTok{(), df\_demand.date\_time.}\BuiltInTok{max}\NormalTok{()))}
\BuiltInTok{print}\NormalTok{(}\StringTok{"Demand Min = }\SpecialCharTok{\{\}}\StringTok{  |  Demand Max = }\SpecialCharTok{\{\}}\CharTok{\textbackslash{}n}\StringTok{"}\NormalTok{.}\BuiltInTok{format}\NormalTok{(}
\NormalTok{    df\_demand.total\_demand.}\BuiltInTok{min}\NormalTok{(), df\_demand.total\_demand.}\BuiltInTok{max}\NormalTok{()))}
\BuiltInTok{print}\NormalTok{(df\_demand.head())}
\BuiltInTok{print}\NormalTok{(}\StringTok{"}\CharTok{\textbackslash{}n}\StringTok{Rows = }\SpecialCharTok{\{\}}\CharTok{\textbackslash{}n}\StringTok{"}\NormalTok{.}\BuiltInTok{format}\NormalTok{(}\BuiltInTok{len}\NormalTok{(df\_demand)))}
\BuiltInTok{print}\NormalTok{(}\StringTok{"Average Datetime frequency: }\SpecialCharTok{\{\}}\CharTok{\textbackslash{}n}\StringTok{"}\NormalTok{.}\BuiltInTok{format}\NormalTok{(}
\NormalTok{    df\_demand.date\_time.diff().mean()))}
\BuiltInTok{print}\NormalTok{(}\StringTok{"NaN values }\CharTok{\textbackslash{}n}\StringTok{{-}{-}{-}{-}{-}{-}{-}{-}{-}{-}}\CharTok{\textbackslash{}n}\SpecialCharTok{\{\}}\StringTok{"}\NormalTok{.}\BuiltInTok{format}\NormalTok{(}
\NormalTok{    df\_demand.isnull().}\BuiltInTok{sum}\NormalTok{().to\_string()))}
\end{Highlighting}
\end{Shaded}

\begin{verbatim}
## Regions = {'NSW1'}
## Date Min = 2010-01-01 00:00:00  |  Date Max = 2021-03-18 00:00:00
## Demand Min = 5074.63  |  Demand Max = 14579.86
##             date_time  total_demand region_id
## 0 2010-01-01 00:00:00       8038.00      NSW1
## 1 2010-01-01 00:30:00       7809.31      NSW1
## 2 2010-01-01 01:00:00       7483.69      NSW1
## 3 2010-01-01 01:30:00       7117.23      NSW1
## 4 2010-01-01 02:00:00       6812.03      NSW1
## 
## Rows = 196513
## Average Datetime frequency: 0 days 00:30:00
## NaN values 
## ----------
## date_time       0
## total_demand    0
## region_id       0
\end{verbatim}

\bigskip

\subsection{Forecast demand dataset}\label{forecast-demand-dataset}

\begin{Shaded}
\begin{Highlighting}[]
\BuiltInTok{print}\NormalTok{(}\StringTok{"Regions = }\SpecialCharTok{\{\}}\StringTok{"}\NormalTok{.}\BuiltInTok{format}\NormalTok{(}\BuiltInTok{set}\NormalTok{(df\_forecast.region\_id)))}
\BuiltInTok{print}\NormalTok{(}\StringTok{"Forecast Date Min = }\SpecialCharTok{\{\}}\StringTok{ | Forecast Date Max = }\SpecialCharTok{\{\}}\StringTok{"}\NormalTok{.}\BuiltInTok{format}\NormalTok{(}
\NormalTok{    df\_forecast.date\_time\_forecast.}\BuiltInTok{min}\NormalTok{(), df\_forecast.date\_time\_forecast.}\BuiltInTok{max}\NormalTok{()))}
\BuiltInTok{print}\NormalTok{(}\StringTok{"Predict Date Min = }\SpecialCharTok{\{\}}\StringTok{ | Predict Date Max = }\SpecialCharTok{\{\}}\StringTok{"}\NormalTok{.}\BuiltInTok{format}\NormalTok{(}
\NormalTok{    df\_forecast.date\_time\_prediction.}\BuiltInTok{min}\NormalTok{(), }
\NormalTok{    df\_forecast.date\_time\_prediction.}\BuiltInTok{max}\NormalTok{()))}
\BuiltInTok{print}\NormalTok{(}\StringTok{"Forecast Demand Min = }\SpecialCharTok{\{\}}\StringTok{ | Forecast Demand Max = }\SpecialCharTok{\{\}}\CharTok{\textbackslash{}n}\StringTok{"}\NormalTok{.}\BuiltInTok{format}\NormalTok{(}
\NormalTok{    df\_forecast.forecast\_demand.}\BuiltInTok{min}\NormalTok{(), df\_forecast.forecast\_demand.}\BuiltInTok{max}\NormalTok{()))}
\BuiltInTok{print}\NormalTok{(df\_forecast.head())}
\BuiltInTok{print}\NormalTok{(}\StringTok{"}\CharTok{\textbackslash{}n}\StringTok{Rows = }\SpecialCharTok{\{\}}\CharTok{\textbackslash{}n}\StringTok{"}\NormalTok{.}\BuiltInTok{format}\NormalTok{(}\BuiltInTok{len}\NormalTok{(df\_forecast)))}
\BuiltInTok{print}\NormalTok{(}\StringTok{"NaN values }\CharTok{\textbackslash{}n}\StringTok{{-}{-}{-}{-}{-}{-}{-}{-}{-}{-}}\CharTok{\textbackslash{}n}\SpecialCharTok{\{\}}\StringTok{"}\NormalTok{.}\BuiltInTok{format}\NormalTok{(df\_forecast.isnull().}\BuiltInTok{sum}\NormalTok{().to\_string()))}
\end{Highlighting}
\end{Shaded}

\begin{verbatim}
## Regions = {'NSW1'}
## Forecast Date Min = 2009-12-30 12:31:49 | Forecast Date Max = 2021-03-17 23:31:33
## Predict Date Min = 2010-01-01 00:00:00 | Predict Date Max = 2021-03-18 00:00:00
## Forecast Demand Min = 4422.46 | Forecast Demand Max = 14736.66
##            id region_id  ...  date_time_forecast  date_time_prediction
## 0  2009123018      NSW1  ... 2009-12-30 12:31:49            2010-01-01
## 1  2009123019      NSW1  ... 2009-12-30 13:01:43            2010-01-01
## 2  2009123020      NSW1  ... 2009-12-30 13:31:36            2010-01-01
## 3  2009123021      NSW1  ... 2009-12-30 14:01:44            2010-01-01
## 4  2009123022      NSW1  ... 2009-12-30 14:31:35            2010-01-01
## 
## [5 rows x 6 columns]
## 
## Rows = 10906019
## NaN values 
## ----------
## id                      0
## region_id               0
## period_id               0
## forecast_demand         0
## date_time_forecast      0
## date_time_prediction    0
\end{verbatim}

\subsection{Weather forecast dataset}\label{weather-forecast-dataset}

\section{Univariate Analysis}\label{univariate-analysis}

We begin the analysis by focusing on the individual distributions and
characteristics of each dataset. This stage provides context on the
seasonal variability of the data.

\subsection{Temperature dataset}\label{temperature-dataset}

Temperature data over time reveals an expected cyclical pattern
corresponding to seasonal changes consistently across all years.

\includegraphics{unsw-ZZSC9020-report-template_files/figure-latex/unnamed-chunk-6-1.pdf}

\bigskip

\noindent Extreme temperature events occurr consistently across the
year, with a lower count (but still significant) in April and November.
Extreme heat days are defined as days with temperatures above the 90th
percentile in the past year, and extreme cold days are defined as days
with temperatures below the 10th percentile in the past year.

\includegraphics{unsw-ZZSC9020-report-template_files/figure-latex/unnamed-chunk-7-3.pdf}

\subsection{Electricity Demand}\label{electricity-demand}

Initial exploration of electricity demand data shows high fluctuations
with an overall downward trend. Comparing with (Figure x), while both
variables are cyclical, it can be observed a more irregular behaviour in
electricity demand compared to temperature.

\includegraphics{unsw-ZZSC9020-report-template_files/figure-latex/unnamed-chunk-8-5.pdf}
\noindent Demand is stable during the weekdays and declines consistently
on the weekends.

\includegraphics{unsw-ZZSC9020-report-template_files/figure-latex/unnamed-chunk-9-7.pdf}
\bigskip

\noindent Higher electricity demand during winter and summer months,
likely due to higher likeliness of extreme temperature events.

\includegraphics{unsw-ZZSC9020-report-template_files/figure-latex/unnamed-chunk-10-9.pdf}

\section{Bivariate Analysis}\label{bivariate-analysis}

In this next section, we will examine for relationships between the
datasets to find if and how temperature affects both the electricity
demand and its forecast. To proceed with this analysis, all the datasets
are merged following the data cleaning processes mentioned in Section
x.x.

\begin{Shaded}
\begin{Highlighting}[]

\CommentTok{\#Forecast\_interval is the number of hours between prediction and it\textquotesingle{}s forecast}
\NormalTok{interval }\OperatorTok{=} \DecValTok{60}\OperatorTok{*}\DecValTok{60} \CommentTok{\#sets the interval in seconds}
\NormalTok{df\_forecast[}\StringTok{"forecast\_interval"}\NormalTok{] }\OperatorTok{=}\NormalTok{ df\_forecast.date\_time\_prediction }\OperatorTok{{-}}\NormalTok{ df\_forecast.date\_time\_forecast}
\NormalTok{df\_forecast.forecast\_interval }\OperatorTok{=}\NormalTok{ df\_forecast.forecast\_interval.}\BuiltInTok{apply}\NormalTok{(}\KeywordTok{lambda}\NormalTok{ x: x.total\_seconds()}\OperatorTok{/}\NormalTok{interval)}

\CommentTok{\#Rounding forecast time to intervals of 30mins to match df\_demand and only have one record where forecast interval is \textasciitilde{}24hrs}
\NormalTok{interval\_min, interval\_max }\OperatorTok{=} \DecValTok{23}\NormalTok{ , }\DecValTok{25} \CommentTok{\#sets a window for forecast periods}
\NormalTok{df\_forecast\_near24hour }\OperatorTok{=}\NormalTok{ df\_forecast.loc[(df\_forecast.forecast\_interval }\OperatorTok{\textgreater{}}\NormalTok{ interval\_min) }\OperatorTok{\&}\NormalTok{ (df\_forecast.forecast\_interval }\OperatorTok{\textless{}}\NormalTok{ interval\_max)]}
\NormalTok{df\_forecast\_near24hour[}\StringTok{"date\_time\_forecast\_rounded"}\NormalTok{] }\OperatorTok{=}\NormalTok{ df\_forecast\_near24hour.date\_time\_forecast.}\BuiltInTok{apply}\NormalTok{(}\KeywordTok{lambda}\NormalTok{ x: x.}\BuiltInTok{round}\NormalTok{(freq}\OperatorTok{=}\StringTok{\textquotesingle{}30min\textquotesingle{}}\NormalTok{))}
\NormalTok{df\_forecast\_near24hour\_1instance }\OperatorTok{=}\NormalTok{ df\_forecast\_near24hour.loc[df\_forecast\_near24hour.groupby(}\StringTok{"date\_time\_forecast\_rounded"}\NormalTok{)[}\StringTok{"forecast\_interval"}\NormalTok{].idxmax()]}

\CommentTok{\#Merge forecast data with demand data}
\NormalTok{df\_forecast\_near24hour\_1instance\_with\_demand }\OperatorTok{=}\NormalTok{ pd.merge(df\_forecast\_near24hour\_1instance, df\_demand, left\_on }\OperatorTok{=} \StringTok{"date\_time\_forecast\_rounded"}\NormalTok{, right\_on }\OperatorTok{=} \StringTok{"date\_time"}\NormalTok{)}
\NormalTok{df\_forecast\_near24hour\_1instance\_with\_demand[}\StringTok{"forecast\_error"}\NormalTok{] }\OperatorTok{=}\NormalTok{ df\_forecast\_near24hour\_1instance\_with\_demand.total\_demand }\OperatorTok{{-}}\NormalTok{ df\_forecast\_near24hour\_1instance\_with\_demand.forecast\_demand}
\NormalTok{df\_merged }\OperatorTok{=}\NormalTok{ pd.merge(}
\NormalTok{      df\_forecast\_near24hour\_1instance\_with\_demand, df\_temperature, }
\NormalTok{      left\_on }\OperatorTok{=} \StringTok{"date\_time\_forecast\_rounded"}\NormalTok{, }
\NormalTok{      right\_on }\OperatorTok{=} \StringTok{"date\_time"}\NormalTok{)}
\NormalTok{df\_merged[}\StringTok{"forecast\_error\_relative"}\NormalTok{] }\OperatorTok{=}\NormalTok{ df\_merged.forecast\_error}\OperatorTok{/}\NormalTok{df\_merged.total\_demand}
\end{Highlighting}
\end{Shaded}

\subsection{Data correlation analysis}\label{data-correlation-analysis}

As per the literature, the plot of temperature against electricity
demand reveals a distinct U-shaped correlation. This pattern reflects
energy usage behavior: demand increases during colder periods due to
heating needs and rises again during hotter periods as cooling systems
are used more intensively. The lowest demand levels generally occur in
temperate conditions where neither heating nor cooling is heavily used.

\includegraphics{unsw-ZZSC9020-report-template_files/figure-latex/unnamed-chunk-12-11.pdf}
\noindent Same pattern emerges when plotting temperature against the
forecasted electricity demand.

\begin{Shaded}
\begin{Highlighting}[]
\CommentTok{\#Temperature vs Forecast Demand}
\NormalTok{coeffs\_forecast }\OperatorTok{=}\NormalTok{ np.polyfit(df\_merged.temperature, df\_merged.forecast\_demand, deg}\OperatorTok{=}\DecValTok{2}\NormalTok{)}
\NormalTok{trendline\_forecast }\OperatorTok{=}\NormalTok{ np.poly1d(coeffs\_demand)}

\NormalTok{x\_trend }\OperatorTok{=}\NormalTok{ np.linspace(df\_merged.temperature.}\BuiltInTok{min}\NormalTok{(), df\_merged.temperature.}\BuiltInTok{max}\NormalTok{(), }\DecValTok{100}\NormalTok{)}
\NormalTok{y\_trend }\OperatorTok{=}\NormalTok{ trendline\_demand(x\_trend)}

\NormalTok{df\_sampled }\OperatorTok{=}\NormalTok{ df\_merged.sample(frac}\OperatorTok{=}\FloatTok{0.1}\NormalTok{, random\_state}\OperatorTok{=}\DecValTok{42}\NormalTok{)}

\NormalTok{plt.figure(figsize}\OperatorTok{=}\NormalTok{(}\DecValTok{6}\NormalTok{, }\DecValTok{4}\NormalTok{))}
\NormalTok{plt.scatter(df\_sampled.temperature, df\_sampled.forecast\_demand, alpha}\OperatorTok{=}\FloatTok{0.5}\NormalTok{, s}\OperatorTok{=}\DecValTok{20}\NormalTok{, label}\OperatorTok{=}\StringTok{\textquotesingle{}Data Points\textquotesingle{}}\NormalTok{)}
\NormalTok{plt.plot(x\_trend, y\_trend, color}\OperatorTok{=}\StringTok{\textquotesingle{}red\textquotesingle{}}\NormalTok{, linewidth}\OperatorTok{=}\DecValTok{2}\NormalTok{, label}\OperatorTok{=}\StringTok{\textquotesingle{}Linear Trendline\textquotesingle{}}\NormalTok{)}

\NormalTok{plt.xlabel(}\StringTok{\textquotesingle{}Temperature\textquotesingle{}}\NormalTok{)}
\NormalTok{plt.ylabel(}\StringTok{\textquotesingle{}Forecasted Electricity Demand\textquotesingle{}}\NormalTok{)}
\NormalTok{plt.title(}\StringTok{\textquotesingle{}Temperature vs Forecasted demand\textquotesingle{}}\NormalTok{)}
\NormalTok{plt.legend()}
\NormalTok{plt.grid(}\VariableTok{True}\NormalTok{)}
\NormalTok{plt.tight\_layout()}
\NormalTok{plt.show()}
\end{Highlighting}
\end{Shaded}

\includegraphics{unsw-ZZSC9020-report-template_files/figure-latex/unnamed-chunk-13-13.pdf}

\subsection{Forecast error}\label{forecast-error}

Forecast error notably increases as temperature increases above mild
temperatures. This suggests are current forecasting models struggle to
predict demand especially under extreme heat temperatures. In the
contrary, forecasts during extreme cold temperatures, while still
showing an increased forecast error, show to be relatively more accurate
than hot temperatures.

\includegraphics{unsw-ZZSC9020-report-template_files/figure-latex/unnamed-chunk-14-15.pdf}

\section{Summary of Key Findings}\label{summary-of-key-findings}

\begin{itemize}
\item
  U-shaped relationship between temperature and electricity demand:
  Electricity demand increases during both extreme cold and extreme heat
  conditions, with the lowest demand observed during temperate
  conditions. This pattern is consistent with expected heating and
  cooling behavior and is evident in both actual and forecasted demand
  data.
\item
  Forecasting models capture the general seasonal trend: Forecasted
  electricity demand shows a similar U-shaped relationship with
  temperature, indicating that the models are broadly aligned with
  seasonal usage patterns.
\item
  Forecast error increases with temperature, especially during extreme
  heat: Forecast accuracy deteriorates significantly at higher
  temperatures, suggesting that current models underperform during
  periods of extreme heat. In comparison, performance during extreme
  cold is better, though still less accurate than under mild conditions.
\item
  Model improvement needed for extreme temperature scenarios: These
  findings highlight the need for enhancing demand forecasting models,
  particularly in handling extreme heat events where demand becomes more
  volatile and difficult to predict.
\end{itemize}

\chapter{Analysis and Results}\label{analysis-and-results}

\chapter{Discussion}\label{discussion}

Put the results you got in the previous chapter in perspective with
respect to the problem studied.

\chapter{Conclusion and Further
Issues}\label{conclusion-and-further-issues}

What are the main conclusions? What are your recommendations for the
``client''? What further analysis could be done in the future?

\bibliographystyle{elsarticle-harv}
\bibliography{references}

\chapter*{Appendix}\label{appendix}
\addcontentsline{toc}{chapter}{Appendix}







\end{document}

